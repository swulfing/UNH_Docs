% Options for packages loaded elsewhere
\PassOptionsToPackage{unicode}{hyperref}
\PassOptionsToPackage{hyphens}{url}
%
\documentclass[
]{article}
\usepackage{amsmath,amssymb}
\usepackage{lmodern}
\usepackage{iftex}
\ifPDFTeX
  \usepackage[T1]{fontenc}
  \usepackage[utf8]{inputenc}
  \usepackage{textcomp} % provide euro and other symbols
\else % if luatex or xetex
  \usepackage{unicode-math}
  \defaultfontfeatures{Scale=MatchLowercase}
  \defaultfontfeatures[\rmfamily]{Ligatures=TeX,Scale=1}
\fi
% Use upquote if available, for straight quotes in verbatim environments
\IfFileExists{upquote.sty}{\usepackage{upquote}}{}
\IfFileExists{microtype.sty}{% use microtype if available
  \usepackage[]{microtype}
  \UseMicrotypeSet[protrusion]{basicmath} % disable protrusion for tt fonts
}{}
\makeatletter
\@ifundefined{KOMAClassName}{% if non-KOMA class
  \IfFileExists{parskip.sty}{%
    \usepackage{parskip}
  }{% else
    \setlength{\parindent}{0pt}
    \setlength{\parskip}{6pt plus 2pt minus 1pt}}
}{% if KOMA class
  \KOMAoptions{parskip=half}}
\makeatother
\usepackage{xcolor}
\usepackage[margin=1in]{geometry}
\usepackage{graphicx}
\makeatletter
\def\maxwidth{\ifdim\Gin@nat@width>\linewidth\linewidth\else\Gin@nat@width\fi}
\def\maxheight{\ifdim\Gin@nat@height>\textheight\textheight\else\Gin@nat@height\fi}
\makeatother
% Scale images if necessary, so that they will not overflow the page
% margins by default, and it is still possible to overwrite the defaults
% using explicit options in \includegraphics[width, height, ...]{}
\setkeys{Gin}{width=\maxwidth,height=\maxheight,keepaspectratio}
% Set default figure placement to htbp
\makeatletter
\def\fps@figure{htbp}
\makeatother
\setlength{\emergencystretch}{3em} % prevent overfull lines
\providecommand{\tightlist}{%
  \setlength{\itemsep}{0pt}\setlength{\parskip}{0pt}}
\setcounter{secnumdepth}{-\maxdimen} % remove section numbering
\usepackage{booktabs}
\usepackage{longtable}
\usepackage{array}
\usepackage{multirow}
\usepackage{wrapfig}
\usepackage{float}
\usepackage{colortbl}
\usepackage{pdflscape}
\usepackage{tabu}
\usepackage{threeparttable}
\usepackage{threeparttablex}
\usepackage[normalem]{ulem}
\usepackage{makecell}
\usepackage{xcolor}
\ifLuaTeX
  \usepackage{selnolig}  % disable illegal ligatures
\fi
\IfFileExists{bookmark.sty}{\usepackage{bookmark}}{\usepackage{hyperref}}
\IfFileExists{xurl.sty}{\usepackage{xurl}}{} % add URL line breaks if available
\urlstyle{same} % disable monospaced font for URLs
\hypersetup{
  pdftitle={RagePackageTest},
  pdfauthor={Sophie Wulfing},
  hidelinks,
  pdfcreator={LaTeX via pandoc}}

\title{RagePackageTest}
\author{Sophie Wulfing}
\date{2023-02-23}

\begin{document}
\maketitle

\hypertarget{multi-month-closures}{%
\section{Multi Month Closures}\label{multi-month-closures}}

\begin{figure}
\centering
\includegraphics{RageandMultiMonthTest_files/figure-latex/multiMoClosures-1.pdf}
\caption{Projections of different lengths of closures. Agrees with the
7\% increase needed for 3 month closures. Double check with other
graph.}
\end{figure}

\newpage

\hypertarget{rage-package-info}{%
\section{Rage Package info}\label{rage-package-info}}

This package also has the following functions I didn't bother with:

\begin{itemize}
\item
  mpm\_collapse which collapses a matrix population model to a smaller
  number of stages
\item
  mpm\_rearrange which rearranges stages of a matrix population model to
  segregate reproductive and non-reproductive stages
\item
  name\_stages allows user to name stages
\item
  perturb\_stochastic calculates stochastic elasticities from a
  time-series of matrix population models and corresponding population
  vectors
\item
  pop\_vectors derives a hypothetical set of population vectors
  corresponding to a time-series of matrix population models
\item
  repro\_stages Identify which stages in a matrix population model are
  reproductive
\item
  shape\_rep Calculate shape of reproduction over age. TBH I just didn't
  get this one
\item
  shape\_surv Calculate shape of survival over age. Same as above
\item
  standard\_stages Identify stages corresponding to different parts of
  the reproductive life cycle
\end{itemize}

Note: Not using the default matU or matR. Double check that

\hypertarget{age-from-stage}{%
\subsection{Age from Stage}\label{age-from-stage}}

Age specific calculations from Rage package

\begin{tabular}{rrrrr}
\toprule
Age & Reproduction & Survivorship & SurvivalProb & MortalityHazard\\
\midrule
1 & 0.0000000 & 1.0000000 & 0.9048003 & 0.1000411\\
2 & 0.0000000 & 0.9048003 & 0.7670599 & 0.2651903\\
3 & 0.0000000 & 0.6940360 & 0.7123393 & 0.3392010\\
4 & 0.1801271 & 0.4943891 & 0.6830319 & 0.3812137\\
5 & 0.4417776 & 0.3376835 & 0.6650848 & 0.4078408\\
\addlinespace
6 & 0.7077949 & 0.2245882 & 0.6533941 & 0.4255748\\
7 & 0.9405289 & 0.1467446 & 0.6455709 & 0.4376202\\
8 & 1.1279077 & 0.0947340 & 0.6402850 & 0.4458419\\
9 & 1.2709528 & 0.0606568 & 0.6367104 & 0.4514404\\
10 & 1.3761398 & 0.0386208 & 0.6343017 & 0.4552305\\
\addlinespace
11 & 1.4513443 & 0.0244972 & 0.6326881 & 0.4577777\\
12 & 1.5039407 & 0.0154991 & 0.6316142 & 0.4594765\\
13 & 1.5400772 & 0.0097895 & NA & NA\\
\bottomrule
\end{tabular}

\newpage

\hypertarget{life-expectancy-and-longevity}{%
\subsection{Life Expectancy and
longevity}\label{life-expectancy-and-longevity}}

Life expectancy applies Markov chain approaches to calculate. Longevity
- (the age x at which survivorship for a synthetic cohort falls below
some critical proportion (in this case 0.01).

Note: Life expectancy and longevity are very different. Wut?

\begin{tabular}{lr}
\toprule
X1 & X2\\
\midrule
Life Expectancy & 4.062737\\
Life Exp Variance & 5.865930\\
longevity & 12.000000\\
\bottomrule
\end{tabular}

\newpage

\hypertarget{life-table}{%
\subsection{Life Table}\label{life-table}}

\begin{itemize}
\tightlist
\item
  x age at the start of the age interval {[}x, x+1)
\item
  Nx The number of individuals alive at age x. The initial number is set
  with radix
\item
  Dx proportion of original cohort dying during the age interval {[}x,
  x+1)
\item
  lx survivorship, defined as the proportion of initial cohort surviving
  to the start of age interval {[}x, x+1)
\item
  dx proportion of original cohort dying in the age interval {[}x, x+1)
\item
  ax The average time survived within the interval by those that die
  during the age interval {[}x, x+1). Assumed to be 0.5
\item
  hx force of mortality (hazard) during the age interval {[}x, x+1)
\item
  qx probability of death during the interval {[}x, x+1) for those
  entering the interval
\item
  px probability of survival for the interval {[}x, x+1) for those
  entering the interval
\item
  Lx total person-years lived during the interval {[}x, x+1)
\item
  Tx total person years lived beyond age x
\item
  ex remaining life expectancy at age x
\item
  mx per-capita rate of sexual reproduction during the interval {[}x,
  x+1)
\item
  lxmx expected number of sexual offspring per original cohort member
  produced during the interval {[}x, x+1)
\end{itemize}

\begin{tabular}{rrrrrrrrr}
\toprule
x & lx & dx & hx & qx & px & ex & mx & lxmx\\
\midrule
0 & 1.0000000 & 0.0951997 & 0.0999577 & 0.0951997 & 0.9048003 & 3.546039 & 0.0000000 & 0.0000000\\
1 & 0.9048003 & 0.2107642 & 0.2636471 & 0.2329401 & 0.7670599 & 2.866532 & 0.0000000 & 0.0000000\\
2 & 0.6940360 & 0.1996469 & 0.3359857 & 0.2876607 & 0.7123393 & 2.585199 & 0.0000000 & 0.0000000\\
3 & 0.4943891 & 0.1567056 & 0.3766632 & 0.3169681 & 0.6830319 & 2.427255 & 0.1801271 & 0.0890529\\
4 & 0.3376835 & 0.1130954 & 0.4022801 & 0.3349152 & 0.6650848 & 2.321617 & 0.4417776 & 0.1491810\\
\addlinespace
5 & 0.2245882 & 0.0778436 & 0.4192659 & 0.3466059 & 0.6533941 & 2.238925 & 0.7077949 & 0.1589624\\
6 & 0.1467446 & 0.0520106 & 0.4307673 & 0.3544291 & 0.6455709 & 2.161373 & 0.9405289 & 0.1380175\\
7 & 0.0947340 & 0.0340773 & 0.4386006 & 0.3597150 & 0.6402850 & 2.073494 & 1.1279077 & 0.1068513\\
8 & 0.0606568 & 0.0220360 & 0.4439266 & 0.3632896 & 0.6367104 & 1.957489 & 1.2709528 & 0.0770919\\
9 & 0.0386208 & 0.0141236 & 0.4475285 & 0.3656983 & 0.6343017 & 1.789093 & 1.3761398 & 0.0531476\\
\addlinespace
10 & 0.0244972 & 0.0089981 & 0.4499474 & 0.3673119 & 0.6326881 & 1.532303 & 1.4513443 & 0.0355539\\
11 & 0.0154991 & 0.0057097 & 0.4515599 & 0.3683858 & 0.6316142 & 1.131614 & 1.5039407 & 0.0233097\\
12 & 0.0097895 & 0.0097895 & 2.0000000 & 1.0000000 & 0.0000000 & 0.500000 & 1.5400772 & 0.0150765\\
\bottomrule
\end{tabular}

\newpage

\hypertarget{perturbation-matrices}{%
\subsection{Perturbation Matrices}\label{perturbation-matrices}}

Perturbs elements within a matrix population model and measures the
response (sensitivity or elasticity) of the per-capita population growth
rate at equilibrium (λ), or, with a user-supplied function, any other
demographic statistic.

\begin{tabular}{rrrr}
\toprule
0.3741554 & 0.1560649 & 0.0344634 & 0.0049345\\
0.4787263 & 0.1996833 & 0.0440955 & 0.0063136\\
2.4286661 & 1.0130279 & 0.2237048 & 0.0320303\\
15.3512859 & 6.4031296 & 1.4139889 & 0.2024572\\
\bottomrule
\end{tabular}

\begin{tabular}{rrrr}
\toprule
0.2399484 & 0.0000000 & 0.0000000 & 0.1342068\\
0.1342067 & 0.0654763 & 0.0000000 & 0.0000000\\
0.0000000 & 0.1342065 & 0.0894979 & 0.0000000\\
0.0000000 & 0.0000000 & 0.1342063 & 0.0682503\\
\bottomrule
\end{tabular}

\newpage

\hypertarget{perturbation-analysis-of-transition-types}{%
\subsection{Perturbation Analysis of transition
types}\label{perturbation-analysis-of-transition-types}}

Calculates the summed sensitivities or elasticities for various
transition types within a matrix population model (MPM), including
stasis, retrogression, progression, fecundity, and clonality.

Sensitivities or elasticities are calculated by perturbing elements of
the MPM and measuring the response of the per-capita population growth
rate at equilibrium (λ), or, with a user-supplied function, any other
demographic statistic.

\begin{itemize}
\tightlist
\item
  stasis The sensitivity or elasticity of lambda to stasis.
\item
  retrogression The sensitivity or elasticity of lambda to
  retrogression.
\item
  progression The sensitivity or elasticity of lambda to progression.
\item
  fecundity The sensitivity or elasticity of lambda to sexual fecundity.
\item
  clonality The sensitivity or elasticity of lambda to clonality.
\end{itemize}

\begin{tabular}{lrr}
\toprule
Measure & Sensitivity & Elasticity\\
\midrule
Stasis & 1.0000007 & 0.4631730\\
Retrogression & NA & NA\\
Progression & 2.9057432 & 0.4026194\\
Fecundity & 0.0049345 & 0.1342068\\
Clonality & NA & NA\\
\bottomrule
\end{tabular}

\hypertarget{perturbation-analysis-of-vital-rates}{%
\subsection{Perturbation analysis of vital
rates}\label{perturbation-analysis-of-vital-rates}}

Perturbs lower-level vital rates within a matrix population model and
measures the response (sensitivity or elasticity) of the per-capita
population growth rate at equilibrium (λ), or, with a usersupplied
function, any other demographic statistic. These decompositions assume
that all transition rates are products of a stage-specific survival term
(column sums of matU) and a lower level vital rate that is conditional
on survival (growth, shrinkage, stasis, dormancy, or reproduction).
Reproductive vital rates that are not conditional on survival (i.e.,
within a stage class from which there is no survival) are also allowed.

\begin{itemize}
\tightlist
\item
  survival sensitivity or elasticity of demog\_stat to survival
\item
  growth sensitivity or elasticity of demog\_stat to growth
\item
  shrinkage sensitivity or elasticity of demog\_stat to shrinkage
\item
  fecundity sensitivity or elasticity of demog\_stat to sexual fecundity
\item
  clonality sensitivity or elasticity of demog\_stat to clonality
\end{itemize}

\begin{tabular}{lrr}
\toprule
Measure & Sensitivity & Elasticity\\
\midrule
Survival & 1.8922055 & 0.9999992\\
Growth & 1.0406220 & 0.2500416\\
Shrinkage & 0.0000000 & 0.0000000\\
Fecundity & 0.0016331 & 0.1342068\\
Clonality & 0.0000000 & 0.0000000\\
\bottomrule
\end{tabular}

\newpage

\hypertarget{plot-life-cycle}{%
\subsection{Plot life cycle}\label{plot-life-cycle}}

look R can do this for you

\hypertarget{age-of-reproductive-maturity}{%
\subsection{Age of reproductive
maturity}\label{age-of-reproductive-maturity}}

\begin{tabular}{lr}
\toprule
X1 & X2\\
\midrule
Prob of reaching mat & 0.0218669\\
Age at Mat & 6.8211804\\
\bottomrule
\end{tabular}

\hypertarget{vital-rates}{%
\subsection{Vital Rates}\label{vital-rates}}

\hypertarget{corresponding-to-separate-demographic-processes}{%
\subsubsection{Corresponding to separate demographic
processes}\label{corresponding-to-separate-demographic-processes}}

Derive mean vital rates corresponding to separate demographic processes
from a matrix population model. Specifically, this function decomposes
vital rates of survival, progression, retrogression, sexual reproduction
and clonal reproduction, with various options for weighting and grouping
stages of the life cycle.

\begin{tabular}{rrrrr}
\toprule
surv & retr & prog & fec & clo\\
\midrule
0.5434124 & 0 & 0.2612287 & 26.7005 & 0\\
\bottomrule
\end{tabular}

\hypertarget{mean-vital-rates}{%
\subsubsection{Mean vital rates}\label{mean-vital-rates}}

\begin{tabular}{lr}
\toprule
X1 & X2\\
\midrule
Rate of Survival & 0.5434124\\
Rate of Growth & 0.2612287\\
Rate of shrinkage & NA\\
Rate of stasis & 0.7387713\\
\bottomrule
\end{tabular}

\hypertarget{survival-independent-rates}{%
\subsection{Survival independent
rates}\label{survival-independent-rates}}

\begin{tabular}{rrrr}
\toprule
0.6958263 & NA & NA & NA\\
0.3041737 & 0.7122376 & NA & NA\\
NA & 0.2877624 & 0.8082501 & NA\\
NA & NA & 0.1917499 & NA\\
\bottomrule
\end{tabular}

\begin{tabular}{rrrr}
\toprule
NA & NA & NA & 80.67898\\
NA & NA & NA & NA\\
NA & NA & NA & NA\\
NA & NA & NA & NA\\
\bottomrule
\end{tabular}

\hypertarget{stage-specific-vital-rates}{%
\subsection{Stage specific vital
rates}\label{stage-specific-vital-rates}}

\begin{tabular}{rrrrr}
\toprule
Stage & Survival & Growth & Stasis & Reproduction\\
\midrule
1 & 0.9048003 & 0.3041737 & 0.6958263 & NA\\
2 & 0.4519657 & 0.2877624 & 0.7122376 & NA\\
3 & 0.4859363 & 0.1917499 & 0.8082501 & NA\\
4 & 0.3309474 & NA & NA & 80.67898\\
\bottomrule
\end{tabular}

\newpage

\hypertarget{misc}{%
\subsection{Misc}\label{misc}}

Demetrius' entropy, Keyfitz's entropy, generation time, net
reproduction, Time to stable stage

Note: method for net reproductive value (R0) from a matrix population
model. The net reproduction value (R0) is the mean number of recruits
produced during the mean life expectancy of an individual. Note: Can
also specify ``start'' for method but if offspring only arise in stage
start, the two methods give the same result.

Note: method for generation time specified as R0, the time required for
a population to increase by a factor of R0 (the net reproductive rate;
Caswell (2001), section 5.3.5). Other options: the average
parent-offspring age difference (Bienvenu \& Legendre (2015)), or the
expected age at reproduction for a cohort (Coale (1972), p.~18-19).

\begin{tabular}{lr}
\toprule
X1 & X2\\
\midrule
Demetrius’ Entropy & 2.1100118\\
Keyfitz’s Entropy & 0.5920757\\
Generation Time & 7.3827459\\
Net Reproduction & 0.8726632\\
Time to Stable Stage Dist & 9.0000000\\
\bottomrule
\end{tabular}

\end{document}
